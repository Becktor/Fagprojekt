%\section{Introduction}
\chapter{Introduction}
The main purpose of the project is to make an advanced game using the Arduino hardware.
Arduino is a programmable piece of hardware. Combining it with the  Gameduino 2 makes
it possible for us to create a rich game. The Gameduino extends the Arduino with a
touch screen, extra space and processing power.
\section{Arduino}
Programming an Arduino is done in its own language, simply named \emph{Arduino programming language}\footnote{http://www.arduino.cc/}. It is based off Wiring, and is reminiscent of both C and C++, but is neither. The IDE is based off the Processing IDE. This mix of languages and sparse documentation makes it difficult for the programmer, even though the Arduino was meant to ease newbies into programming. An example, would be the keyword {\tt new}, which officially is not supported\footnote{The official reference page http://arduino.cc/en/Reference/HomePage does not include the keyword.}. This is confusing, because new is completely functional, even syntax highlit. It instantiates an object in the heap rather than the stack, and returns the pointer to this object. Other unlisted keywords we found during development are delete and clear.\\ %Carsten
Our arduinos and their technicalities
\section{Gameduino 2}
The gameduino and technicalities.