\chapter{Introduction}
This goal of this project is a to implement a simple platforming game on the Arduino Leonardo with the Gameduino 2. The game will be akin to arcade platforming games like Castlevania and Spelunky, though a lot more simple. It should feature competent enemies, a scoring system and a random level generator, as well as usual platforming mechanics.
\newline
The point is to attempt creating an advanced game compared to the examples on Gameduino 2's official example page. The Arduino is a very limited environment to work with, so the project will focus on writing efficient but still maintainable code.

\section{Arduino} %Carsten
An Arduino is a tiny programmable microprocessor which can be modified or interfaced with other pieces of hardware. Its main use was by hobbyists, since it can be used to create a very wide array of devices. Programming an Arduino is done in its own language, simply named \emph{Arduino programming language}\footnote{http://www.arduino.cc/}. It is based off Wiring, and is reminiscent of both C and C++, but is neither. The IDE is based off the Processing IDE. There are many kinds of Arduinos. The ones we used in this project is the Duemilanove and the Leonardo.

\section{Gameduino 2}
The Gameduino 2, a kickstarted successor to the Gameduino, is a piece of hardware made for the Arduino. Its use is for making games on the Arduino platform, as the name suggests. It features a small touchscreen, an accelerometer, a GPU, sound output and a slot for a SD card.
\newline
The Gameduino 2 is very new, and very little documentation or user created content exists. This could be an issue during development, so we will have to solve problems ourselves most of the time.