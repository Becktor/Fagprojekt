\chapter{Introduction}
The goal of this project is to implement a simple platforming game on the Arduino Leonardo using the Gameduino 2 shield. The game will be akin to arcade platforming games like Castlevania and Spelunky, though a lot more simple. It should feature competent enemies, a scoring system and a random level generator, as well as standard platforming mechanics.
\newline
The point is to create an more advanced game compared to the examples on Gameduino 2's official example page. The Arduino is a very limited environment to work with, so the project will focus on writing efficient yet maintainable code.

\section{Arduino} %Carsten
An Arduino is a tiny programmable microprocessor which can be modified or interfaced with other pieces of hardware. It's tailored to programming hobbyists, since it can be used to create a wide array of devices. Programming an Arduino is done in its own language, simply named \emph{Arduino programming language}\footnote{http://www.arduino.cc/}. It is based off \emph{Wiring}, and is reminiscent of both C and C++. The IDE is based off the \emph{Processing} IDE. There are many kinds of Arduinos, and the ones we used in this project are the \emph{Duemilanove} and the \emph{Leonardo}.

\section{Gameduino 2}
The Gameduino 2 is a kickstarted successor to the Gameduino. It's a piece of hardware made to make games on the Arduino platform. It features a small touchscreen, an accelerometer, a GPU, sound output, and a slot for an SD card. The Gameduino 2 is very new, and as such, little documentation or user created content exists.