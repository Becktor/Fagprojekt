
\chapter{Problem Analysis}
We aim to implement a game on the Arduino which runs reasonably well with the following features:
\begin{itemize}
\item Solid platforming mechanics
\item Competent enemies
\item Randomly generated maps
\end{itemize}
The usual platformer mechanics is moving and jumping. The movement should feel responsive and natural, which would include some simple physics engine to conserve momentum. The enemies should prove a challenge to the player, but also be more advanced than simply walking left to right. They should have different behaviors and be able to hunt the player. The random maps should be different enough to provide a varied gameplay experience. The enemies should be randomly placed placed around the map
\newline
In addition to the main goals of the project, we have a range of side goals: Scoring, graphics and sounds.
\newline
To give the game a purpose, we want to add a scoring system, with high-scores hat are saved between sessions. This would give the player a goal, without creating some kind of ending. The player would earn points by killing enemies and collecting coins around the map, motivating the player to explore the map.
\newline
We want simple yet easily recognizable imagery. Animations should include smooth transitions and it should be easy for the player to feel like he or she is truly in control of their character's actions in the game. There shouldn't be any points in gameplay where it's confusing as to what exactly is happening on screen. Essentially the game should look nice but still keep the sense of retro genre game.
\newline
And finally, to use more of the Gameduino 2's features, we would like to add sounds. They would help adding feedback to the player, like when attacking or collecting coins. This is an easy way to give the player some response in a game instead of a lot of effects. It would be easier to code and take less space.
