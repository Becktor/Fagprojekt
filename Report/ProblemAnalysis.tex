\chapter{Problem Analysis and Requirements specification}
None of us had experience with coding in a unit with a 'simple' microprocesser. On top of that we had never coded in C++, but we had some experience in C.

///////
We were sure that the limited power and space capacity would be a challange. Another thing was its IDE. The Arduino IDEs language is not the most
sophisticated language according to what we heard. We did neither have any kind of emulator, which we definitely would have used, before getting the unit in our hands.
////////////

We were also sceptical about what Gameduino 2 was capable of. The examples
which were shared through their site\footnote{http://excamera.com/sphinx/gameduino2/} did not look impressing. Only one of them
looked like an advanced game.

As we did not have any experience with these units, we were also unsure of how to connect them. Do we need to do any soldering? or but any extra equipment?

\newpage

\section{Gameplay}%Jonathan Philiph
///////
We decided to create an arcade game with focus on increasing difficulty per level and having a high score as the intensive to play the game.

We want simple yet easily recognizable imagery. Animations should include smooth transitions and it should be easy for the player to feel like he or she is truly in control of their character's actions in the game. There shouldn't be any points in gameplay where it's confusing as to what exactly is happening on screen. Essentially the game should look nice but still keep the sense of retro genre game.
 
High scores is something you almost always see in arcade games or just smaller games. Its a great way to compare and compete and to show who's the best.

Coin were added as an additional game play element to broaden the game. The player now has an incentive to go explore the entirety of the map, since collecting coins is an easy way to get additional score.
//////////


