\chapter{Discussion}
 
One of our biggest problems here towards the end has been the amount of flash memory. This set a limit for what we could add and how we could add it. Some places in our code we have had to implement some things rather crudely to save flash memory, an example is that we have not used any getters/setters all is done with public variables. We have also had problems the amount we could store on our GD2 file this limited us to a few selected sprites and audio files.

Furthermore we have had some issues with spawning coins.\\

Additional features could be implemented but that would require a board with more memory. If this was the case we could implement the features we wanted originally, story, more monsters, items, levels and so on.\\
We have been very limited by the boards capacities. But this has taught us to work with code another way than we usually do. Today when making a platformer for computer we will never reach a point where we have too much code, but how compact is the code, how optimized is it? instead of optimizing you would just build additional features once the last feature worked. Since none of us had experience with c++ we have struggled a bit with the documentation on the arduino site and the "guide book" from the GD2 library. We have had some cases where the documentation has been plain wrong, and mostly confusing and badly written with bad examples.


\section{Meeting Requirements}

\subsection*{Cut content}
Items, merchant and levels
increasing difficulty?
Multiple enemies?
\subsection*{Additional content}
More or less advanced physics engine.
Coin physics.
