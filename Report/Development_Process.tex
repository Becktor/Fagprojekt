%Discuss how development processed, problems encountered and how some features were cut or added.
\chapter{Development Process}
Work schedule during 13 week period.\\
       during 3 week period.
       \section{Timeplan}%Cebrail/Jonathan

       There are several time planning models in the software world. There is agile, iterative
       and incremental.

       When we started the `Fagprojekt' we created a waterfall chart containing our plans
       for every week, it was structured in a waterfall chart. The waterfall system is a
       sequential design process.
       It is designed to get through the project phases and have a product as soon as
       possible. The phases in our project can be seen in the figure below.

       As we revisited our waterfall model steps over time, our main
       time plan model can be considered to be iterative. The revisits have mainly
       been to extend features,debugging or optimizing.

       When the waterfall ends and we still have time
       we will go back and visit the steps and check for new requirements.
       \\The waterfall gives
       a good picture of the big phases, but the pre-planned week schedules are
       not always much help, as they are not dynamic. We can't reconstruct our waterfall every
       time we meet a conflict. This is here where the timeboxes are handy, which is used
       to the most detailed part of the time planning - see next subsection.
       You can check our waterfall timeplan in the appendices, please
       see Figure~\ref{fig:Waterfall_chart} for that.

       \begin{figure}[h]
       \centering
       \includegraphics[scale=0.6]{Figures/Waterfall}
       \caption{An overview of the waterfall phases.}
       \label{fig:Waterfall}
       \end{figure}

       \newpage
       \subsection{Time Boxing} %Jonathan
       We were conviced that using "Time Boxing" would be the way to go.
       Timeboxing divides The schedule into a number of separate time periods(timeboxes), with each part having its own deliverables, deadline and budget.
       Breaking bigger tasks into smaller tasks with better manageable time frames.
       What also is important is that by the end of each timebox we need to have a product that if all else fails we can roll back and release our game from an earlier state. The following table shows the timeboxes we have created during the project.

       % Please add the following required packages to your document preamble:
       % \usepackage[table,xcdraw]{xcolor}
       % If you use beamer only pass "xcolor=table" option, i.e. \documentclass[xcolor=table]{beamer}

       \begin{table}[h]
       \begin{tabular}{llll}
       \rowcolor[HTML]{BBDAFF}
{\color[HTML]{000000} \textbf{week 8-10}} & {\color[HTML]{000000} \textbf{week 11-13}} & {\color[HTML]{000000} \textbf{week 14-15}} & {\color[HTML]{000000} \textbf{week 16-17}} \\
    Code exercise                        & Enemies                               & Scene generation (simple)             & Graphics                              \\
    Game design                          & Collision Detection                   & Player                                & Sprites                               \\
    Class Design                         & AI                                    &                                       & Map generation                        \\
    & Input                                 &                                       &                                       \\
    Report                               & Report                                & Report                                & Report                                \\
    &                                       &                                       &                                       \\
    \rowcolor[HTML]{BBDAFF} 
    \textbf{week 18-90}                       & \textbf{week 19}                           & \textbf{week 20}                           & \textbf{week 22-25}                        \\
      Endgame                              & Scene generation                      & Sprites                               & Optimization                          \\
      Map generation                       & Optimization                          & Sound                                 &                                       \\
      & Attack                                & Optimization                          &                                       \\
      &                                       & Animation                             &                                       \\
      Report                               & Report                                & Report                                &                                      
      \end{tabular}
      \caption{The timeboxes are seperated per week}
      \end{table}
      \newpage
      \section{Hardware}
      Recieving the Arduino.\\
        Old arduino.\\
        Arduino burnout.

        \subsection{Nunchuck}
        Nunchuck compatability.\\
